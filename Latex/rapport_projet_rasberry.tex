\documentclass[a4paper,10pt]{article}

\title{Serveur domotique}

\author{TERRIE Corentin \\ CROS Bastien \\ MONNIER Matthias}

\date{\today}

\usepackage[utf8]{inputenc}
\usepackage[french]{babel} 
\usepackage{lmodern} % Pour changer le pack de police
\usepackage{makeidx}
\usepackage{fancyhdr}
\usepackage{graphicx}
\usepackage[lofdepth,lotdepth]{subfig}
\usepackage{float}
\usepackage{hyperref}
%---------------------------------------------------------
\usepackage{listings}
\usepackage{textcomp}
% JAVA en couleur ;-)- -----------------------------------
%\lstset{
%language=Java,
%basicstyle=\normalsize, % ou ça==> basicstyle=\scriptsize,
%upquote=true,
%aboveskip={1.5\baselineskip},
%columns=fullflexible,
%showstringspaces=false,
%extendedchars=true,
%breaklines=true,
%showtabs=false,
%showspaces=false,
%showstringspaces=false,
%identifierstyle=\ttfamily,
%keywordstyle=\color[rgb]{0,0,1},
%commentstyle=\color[rgb]{0.133,0.545,0.133},
%stringstyle=\color[rgb]{0.627,0.126,0.941},
%}
%----------------------------------------------------------

\begin{document}
\makeatletter
  \begin{titlepage}
  \centering
      {\Large \textsc{École Universitaire Polytechnique de Montpellier}}\\
      \textsc{Microélectronique et Automatique - Électronique et Informatique Industrielle }\\
    \noindent\hrulefill 
    \\
    \vspace{2 cm}
      {\large	\@date\\}
    \vspace{2cm}
       {\huge \textbf{\@title}} \\
	
	\vspace{1cm}
      %\includegraphics[scale=0.15]{images/head.png}\\

    \vspace{2em}
        {\large \@author} \\
    
    \vspace{2.6cm}
        %\includegraphics[height=0.15\textheight]{images/mea.jpg}
        \hfill
        %\includegraphics[height=0.15\textheight]{images/um.png}
  \end{titlepage}
\makeatother
\tableofcontents
\clearpage


%-------------------------------------------------------------------------------

\section{Partie Android (client)}



%-------------------------------------------------------------------------------------------------------------

\section{Partie Raspberry (serveur)}



%-------------------------------------------------------------------------------------------------------------

\section{JSON (JavaScript Object Notation)}

Nous avons rencontré un problème lors de l'envoi de données entre la tablette et la Raspberry. En effet les données n'ont pas le même format, elles ne sont pas codées sur le même nombre d'octets sur l'une et l'autre des platformes. Pour résoudre ce problème nous avons décidé d'utiliser un format de donnée spécifique : le JSON.

JSON est un format de données textuelles dérivé de la notation des objets du langage JavaScript. Il permet de représenter de l’information structurée comme le permet XML par exemple. 

Voici un exemple d'objet JSON :
\begin{figure}[H]
\centering\includegraphics[scale=0.7]{images/json_exemple.png}
\caption{Exemple d'objet JSON.}
\end{figure}

JSON permet donc de représenter n'importe quelles données concrètes et surtout les représenter dans n'importe quel langage de programmation. Ses autres avantages sont :
\begin{itemize}
\item peu verbeux donc très lisible
\item facile à apprendre de par sa syntaxe simple
\item ses types de données sont connus et simples à décrire
\end{itemize}

\subparagraph{Sous Android} La plate-forme Android inclut les bibliothèques json.org qui permettent de travailler facilement avec des fichiers JSON. Elle contient un parser et l'utilisation est plutôt simple. Nous avons utilisé \href{http://vogella.developpez.com/tutoriels/android/utilisation-format-json/}{ce tutoriel}.
\subparagraph{Sur la Raspberry} Nous avons opté pour un parser écrit en C. Il s'appelle \textbf{jsmn} (à prononcer "jasmine") et est très minimaliste. Il est fait exprès pour les systèmes embarquée à ressources limitées. Il est très simple d'utilisation et c'est pourquoi nous l'avons choisi. Voici le \href{http://zserge.com/jsmn.html}{liens} contenant toutes les informations.

%-------------------------------------------------------------------------------------------------------------
\section{Protocoles domotiques}

\subsection{Les protocoles grand public}
%-----
\paragraph{X10}
Le X10 est un vieux protocole (développé en 1975) par courant porteurs.  Les modules X10 peuvent être piloté par des télécommandes radio (433MHz). 

%\underline{\textbf{Les plus :}}
\subparagraph{Les plus :}
\begin{itemize}
\item Protocole le moins cher dans le domaine des automatismes résidentiels
\item Communauté d'utilisateurs très active
\item Bonne distribution des produits
\end{itemize}
\subparagraph{Les moins :}
\begin{itemize}
\item Gros problème de sécurité (toute personne possédant l'accès à une partie de l'installation électrique peut envoyer des ordres X10)
\item Incompatibilité entre les réseaux électriques des différents pays
\item Pas de retour d'état des modules et collisions non gérées
\end{itemize}
Le X10 est vieillissant et par conséquent de nombres constructeurs le délaissent. Il comporte comme seul avantage d'être simple et n'est donc plus conseillé.
%-----
\paragraph{OREGON}
\paragraph{OREGON}


%-------------
\subsection{Les protocoles de rupture technologique}
Deux protocoles seront détaillés dans cette partie : Z-Wave et EnOcean. Ils partagent de nombreuses similitudes, ils sont tous deux :
\begin{itemize}
\item des protocoles domotiques
\item des technologies sans fil
\item des technologies disponibles via plusieurs centrales domotiques
\item des technologies mises en œuvre par plusieurs constructeurs de périphériques.

\end{itemize}
%-----
\paragraph{Z-Wave}
Le Z-Wave est un protocole radio conçu pour la domotique. Il utilise la fréquence $848.42$MHz.
Ses caractéristiques sont qu'il est relativement sécurisé, à double sens (chaque composant est à la fois récepteur et émetteur) et qu'il utilise un réseau maillé.
%\underline{\textbf{Les plus :}}
\subparagraph{Les plus :}
\begin{itemize}
\item Grande richesse fonctionnelle.
\item Topologie maillée du réseau sans fil (contrainte de portée levée car les périphériques relayent les informations entre eux).
\item Retour d'état des périphérique (acquittement de commande, retour d'état de la batterie etc...).
\item Grande liberté de choix sur les périphériques.
\end{itemize}
\subparagraph{Les moins :}
\begin{itemize}
\item Complexité de la mise en place.
\item Prix élevé des modules.
\item Consommateur de pile.\newline
\end{itemize}
Le Z-Wave est un protocole de choix pour les actionneurs mais moins pour les capteurs. Voici le résumé graphique :

\begin{figure}[H]
\centering\includegraphics[scale=0.7]{images/forces-protocole-zwave.jpg}
\caption{Forces et faiblesses du protocole Z-Wave.}
\end{figure}

%-----
\paragraph{EnOcean}
Le Enocean est également un protocole radio utilisant la fréquence $848.42$MHz. Ce protocole à l'avantage d'être sans fils et sans piles. En effet les périphériques puisent leur source d'énergie de leur environnement direct. Il utilise notamment l'effet photovoltaïque (transforme la lumière en électricité), l'effet piezo-électrique (transforme un choc ou une forte pression en électricité) ainsi que l’effet Peltier ou effet thermoélectrique (transforme une différence de température constaté à un instant donné en électricité).

Lorsqu’un interrupteur EnOcean utilise des cristaux piezo-électrique pour produire son électricité, il communique son ordre ON/OFF par voie hertzienne en utilisant l’énergie fournie par la pression mécanique de celui qui actionne l’interrupteur. Il peut ainsi générer le peu de courant nécessaire pour envoyer l’information à la centrale.

%\underline{\textbf{Les plus :}}
\subparagraph{Les plus :}
\begin{itemize}
\item Flexibilité élevée lors de la mise en oeuvre mais également en cas de modification de l'installation.
\item Pas besoin de changer les piles.
\item Pas de coût de consommables.
\end{itemize}
\subparagraph{Les moins :}
\begin{itemize}
\item Coût des modules plus important que certaines autres technologies.
\item Design peut attrayant de certains modules.
\item Retour d'état non disponible pour certains périphériques (seuls ceux ayant une batterie de stockage de l'énergie le peuvent).
\item Le réseaux n'est pas maillé. Mais il existe des relayeurs graces auxquels le réseaux peut rayonner sur une portée de 200m.\newline
\end{itemize}


Voici le résumé graphique :
\begin{figure}[H]
\centering\includegraphics[scale=0.7]{images/forces-protocole-enocean.jpg}
\caption{Forces et faiblesses du protocole EnOcean.}
\end{figure}

Les deux graphiques précédent sont issues du site \url{www.abavala.com}. Elles ne sont pas issues de mesures scientifiques mais d’avis personnels exprimés librement. Voici le graphique qui regroupes ces deux protocoles :

\begin{figure}[H]
\centering\includegraphics[scale=0.7]{images/forces-enocean-vs-z-wave.jpg}
\caption{Comparaison des deux protocoles.}
\end{figure}
%-------------------------------------------------------------------------------------------------------------

\section{Annexes}
http://vogella.developpez.com/tutoriels/android/utilisation-format-json/

 
%-------------------------------------------------------------------------------------------------------------

\end{document}
